% IEEE Conference Paper Template
% Auditory Feedback Mechanisms for Monitoring Symplectic Conservation 
% in Molecular Dynamics Simulations

\documentclass[conference]{IEEEtran}

% Packages
\usepackage{cite}
\usepackage{amsmath,amssymb,amsfonts}
\usepackage{algorithmic}
\usepackage{graphicx}
\usepackage{textcomp}
\usepackage{xcolor}
\usepackage{booktabs}
\usepackage{multirow}
\usepackage{hyperref}
\usepackage{listings}
\usepackage{physics}

% Code listing style
\lstset{
    basicstyle=\footnotesize\ttfamily,
    breaklines=true,
    frame=single,
    language=Python
}

\def\BibTeX{{\rm B\kern-.05em{\sc i\kern-.025em b}\kern-.08em
    T\kern-.1667em\lower.7ex\hbox{E}\kern-.125emX}}

\begin{document}

\title{Auditory Feedback Mechanisms for Monitoring Symplectic Conservation in Molecular Dynamics Simulations\\
{\footnotesize A Comparative Study of Numerical Integration Methods with Real-Time Sonification}
}

\author{\IEEEauthorblockN{Ryan Kamp}
\IEEEauthorblockA{\textit{Department of Computer Science} \\
\textit{University of Cincinnati}\\
Cincinnati, OH, USA \\
kamprj@mail.uc.edu \\
https://github.com/ryanjosephkamp}
\and
\IEEEauthorblockN{}
\IEEEauthorblockA{\textit{Week 1 Project 1} \\
\textit{Biophysics Portfolio}\\
January 21, 2026}
}

\maketitle

\begin{abstract}
Numerical integration of Hamiltonian systems lies at the foundation of molecular dynamics (MD) simulations, yet the choice of integration algorithm profoundly impacts long-term simulation stability and physical validity. This work presents a novel auditory feedback system that sonifies energy conservation violations in real-time, providing immediate sensory feedback on integrator performance. We implement and compare four numerical integration schemes---Forward Euler, fourth-order Runge-Kutta (RK4), Velocity Verlet, and Leapfrog---applied to canonical Hamiltonian test systems. Our results demonstrate that symplectic integrators (Velocity Verlet, Leapfrog) exhibit bounded energy oscillations suitable for arbitrarily long simulations, while non-symplectic methods (Forward Euler, RK4) display systematic energy drift that manifests as audible pitch and amplitude changes. The sonification framework transforms abstract numerical errors into visceral auditory experiences, serving both as an educational tool and a novel debugging mechanism for computational physics applications.
\end{abstract}

\begin{IEEEkeywords}
symplectic integrators, molecular dynamics, numerical methods, sonification, energy conservation, Hamiltonian mechanics, Velocity Verlet, computational biophysics
\end{IEEEkeywords}

\section{Introduction}

\subsection{Background and Motivation}

Molecular dynamics (MD) simulations have become indispensable tools in computational biology, materials science, and drug discovery, enabling researchers to study atomic-scale phenomena at temporal resolutions inaccessible to experimental techniques \cite{karplus2002}. At the heart of every MD simulation lies a numerical integrator---an algorithm that propagates the positions and velocities of particles forward in time according to Newton's equations of motion.

The choice of numerical integrator is not merely a computational detail but a fundamental decision that determines whether a simulation produces physically meaningful results. For Hamiltonian systems---those described by a Hamiltonian function $H(q, p)$ representing the total energy---the exact dynamics preserve several geometric properties, most notably the conservation of total energy and the preservation of phase space volume (Liouville's theorem) \cite{arnold1989}.

Standard numerical methods from applied mathematics, such as the explicit Euler method or Runge-Kutta schemes, are designed for general ordinary differential equations (ODEs) and do not respect these geometric invariants. When applied to Hamiltonian systems, they introduce systematic errors that accumulate over time, leading to unphysical behavior such as energy drift, phase space distortion, and simulation ``explosions.''

In contrast, \textbf{symplectic integrators} are specifically designed to preserve the symplectic structure of Hamiltonian phase space. While they may introduce bounded oscillations in energy, they never exhibit systematic drift, making them suitable for simulations spanning millions of timesteps \cite{leimkuhler2004}.

\subsection{The Sonification Approach}

Traditional methods for assessing integrator quality rely on post-hoc analysis of energy time series or phase space plots. We propose an alternative paradigm: \textbf{real-time auditory sonification} of energy conservation.

The human auditory system is remarkably sensitive to changes in pitch, amplitude, and timbre \cite{bregman1990}. By mapping energy values to auditory parameters, we create an immediate feedback loop where stable energy produces a steady tone, growing energy causes rising pitch and volume, and fluctuating energy introduces distortion.

\subsection{Objectives}

The objectives of this work are to:
\begin{enumerate}
    \item Implement four numerical integrators from first principles
    \item Apply these integrators to canonical Hamiltonian test systems
    \item Develop a real-time sonification engine mapping energy to audio
    \item Quantitatively compare integrator performance
    \item Demonstrate the educational value of auditory feedback
\end{enumerate}

\section{Theoretical Background}

\subsection{Hamiltonian Mechanics}

A Hamiltonian system is defined by a scalar function $H(q, p, t)$, the Hamiltonian, which typically represents the total energy. The canonical coordinates $q$ (positions) and $p$ (momenta) evolve according to Hamilton's equations:

\begin{equation}
\frac{dq}{dt} = \frac{\partial H}{\partial p}, \quad \frac{dp}{dt} = -\frac{\partial H}{\partial q}
\end{equation}

For autonomous systems, the Hamiltonian is conserved:

\begin{equation}
\frac{dH}{dt} = \frac{\partial H}{\partial q}\frac{\partial H}{\partial p} - \frac{\partial H}{\partial p}\frac{\partial H}{\partial q} = 0
\end{equation}

\subsection{Symplectic Structure}

The phase space of a Hamiltonian system possesses a geometric structure characterized by the symplectic 2-form $\omega = \sum_i dq_i \wedge dp_i$. A transformation $(q, p) \to (Q, P)$ is \textbf{symplectic} if it preserves this 2-form. Equivalently, the Jacobian matrix $M$ satisfies:

\begin{equation}
M^T J M = J, \quad \text{where } J = \begin{pmatrix} 0 & I \\ -I & 0 \end{pmatrix}
\end{equation}

The exact time evolution of a Hamiltonian system is a symplectic map. Liouville's theorem---that phase space volume is preserved---is a direct consequence of this symplectic structure \cite{marsden1999}.

\subsection{The Harmonic Oscillator}

The simple harmonic oscillator provides an ideal test case. The Hamiltonian is:

\begin{equation}
H = \frac{1}{2}mv^2 + \frac{1}{2}kx^2
\end{equation}

where $m$ is mass and $k$ is the spring constant. The natural frequency is $\omega = \sqrt{k/m}$. The analytical solution is:

\begin{equation}
x(t) = A\cos(\omega t + \phi), \quad v(t) = -A\omega\sin(\omega t + \phi)
\end{equation}

The phase space trajectory is an ellipse, and total energy is exactly conserved.

\section{Numerical Integration Methods}

\subsection{Forward Euler Method}

The Forward Euler method uses current derivatives to estimate the next state:

\begin{equation}
x_{n+1} = x_n + v_n \Delta t, \quad v_{n+1} = v_n + a_n \Delta t
\end{equation}

\textbf{Properties:} Order 1, non-symplectic, energy grows exponentially.

The Jacobian of the map has determinant $\det(M) = 1 + \omega^2 \Delta t^2 > 1$, indicating phase space expansion at each step.

\subsection{Fourth-Order Runge-Kutta (RK4)}

RK4 achieves fourth-order accuracy by evaluating derivatives at multiple points:

\begin{align}
k_1 &= f(t_n, y_n) \\
k_2 &= f(t_n + \tfrac{\Delta t}{2}, y_n + \tfrac{\Delta t}{2}k_1) \\
k_3 &= f(t_n + \tfrac{\Delta t}{2}, y_n + \tfrac{\Delta t}{2}k_2) \\
k_4 &= f(t_n + \Delta t, y_n + \Delta t \cdot k_3) \\
y_{n+1} &= y_n + \tfrac{\Delta t}{6}(k_1 + 2k_2 + 2k_3 + k_4)
\end{align}

\textbf{Properties:} Order 4, non-symplectic, slow energy drift.

\subsection{Velocity Verlet Method}

The Velocity Verlet algorithm \cite{verlet1967} is symplectic and second-order:

\begin{align}
x_{n+1} &= x_n + v_n \Delta t + \tfrac{1}{2}a_n \Delta t^2 \\
a_{n+1} &= a(x_{n+1}) \\
v_{n+1} &= v_n + \tfrac{1}{2}(a_n + a_{n+1})\Delta t
\end{align}

\textbf{Properties:} Order 2, symplectic, time-reversible, bounded energy oscillations.

\subsection{Leapfrog Method}

The Leapfrog method uses staggered time points (kick-drift-kick):

\begin{align}
v_{n+1/2} &= v_n + \tfrac{1}{2}a_n \Delta t \\
x_{n+1} &= x_n + v_{n+1/2} \Delta t \\
v_{n+1} &= v_{n+1/2} + \tfrac{1}{2}a_{n+1} \Delta t
\end{align}

Mathematically equivalent to Velocity Verlet with identical properties.

\section{Audio Sonification Methodology}

\subsection{Energy-to-Frequency Mapping}

We employ a logarithmic mapping for perceptual linearity:

\begin{equation}
f(E) = f_0 + \frac{f_{\text{range}}}{2} \cdot \log_2\left(\frac{E}{E_0}\right)
\end{equation}

where $f_0 = 220$ Hz (A3), $f_{\text{range}} = 440$ Hz, and $E_0$ is the initial energy.

\subsection{Energy-to-Amplitude Mapping}

Volume is mapped linearly:

\begin{equation}
A(E) = A_0 \cdot \frac{E}{E_0}
\end{equation}

\subsection{Distortion for Instability}

Energy fluctuations introduce distortion via soft clipping:

\begin{equation}
s_{\text{distorted}} = \frac{\tanh(g \cdot s)}{g}, \quad g = 1 + 5D
\end{equation}

where $D$ is the coefficient of variation of recent energy values.

\section{Experimental Setup}

\subsection{Configuration}

\begin{itemize}
    \item Mass: $m = 1.0$ kg, Spring constant: $k = 1.0$ N/m
    \item Initial conditions: $x_0 = 1.0$ m, $v_0 = 0.0$ m/s
    \item Time step: $\Delta t = 0.02$ s, Duration: 10.0 s
    \item Initial energy: $E_0 = 0.5$ J
\end{itemize}

\subsection{Metrics}

\begin{itemize}
    \item Energy drift: $(E_{\text{final}} - E_0)/E_0 \times 100\%$
    \item Maximum deviation: $\max_n |E_n - E_0|/E_0 \times 100\%$
    \item Energy standard deviation: $\sigma_E$
\end{itemize}

\section{Results}

\subsection{Quantitative Comparison}

Table~\ref{tab:results} presents energy conservation metrics for all integrators.

\begin{table}[htbp]
\caption{Energy Conservation Metrics}
\label{tab:results}
\centering
\begin{tabular}{lcccc}
\toprule
\textbf{Integrator} & \textbf{Sympl.} & \textbf{Drift (\%)} & \textbf{Max Dev (\%)} \\
\midrule
Forward Euler & No & +22.135 & 22.135 \\
Runge-Kutta 4 & No & $<$0.001 & 0.001 \\
Velocity Verlet & Yes & $-$0.003 & 0.020 \\
Leapfrog & Yes & $-$0.003 & 0.020 \\
\bottomrule
\end{tabular}
\end{table}

\subsection{Phase Space Analysis}

Forward Euler exhibits spiral outward motion due to phase space expansion ($\det(M) > 1$). Symplectic integrators maintain closed orbits with only slight thickness from bounded oscillations.

\subsection{Sonification Results}

\begin{itemize}
    \item \textbf{Forward Euler:} Pitch rises from 220 Hz to $\sim$318 Hz over 10 seconds---an immediately audible change.
    \item \textbf{Velocity Verlet:} Steady tone at 220 Hz throughout.
\end{itemize}

The auditory difference is immediately apparent, demonstrating the educational value of sonification.

\section{Discussion}

\subsection{Importance of Symplecticity}

Our results underscore why symplectic integrators are the universal standard in MD. Forward Euler's 22\% energy drift over 1.6 periods would lead to catastrophic failure in long simulations. Even RK4's small drift accumulates over $10^9$ timesteps.

Velocity Verlet, despite being only second-order, preserves the essential geometric structure. This explains its universal adoption in MD packages (GROMACS, AMBER, OpenMM, LAMMPS, NAMD).

\subsection{Sonification as Education}

Users with no numerical methods background could immediately identify problems with Forward Euler based on rising pitch alone. Applications include:
\begin{itemize}
    \item Physics education (demonstrating conservation laws)
    \item Debugging (auditory monitoring of simulations)
    \item Accessibility (conveying simulation health aurally)
\end{itemize}

\subsection{Limitations}

Current limitations include single degree of freedom, real-time constraints, and perceptual limits for slow drift. Future work could explore multi-channel audio and machine learning for anomaly detection.

\section{Conclusions}

We demonstrated the critical importance of symplectic integrators through quantitative analysis and auditory sonification:

\begin{enumerate}
    \item Symplectic integrators exhibit \textbf{bounded energy oscillations} suitable for arbitrarily long simulations.
    \item Forward Euler is \textbf{fundamentally unsuitable} for MD, with 22\% drift over 1.6 periods.
    \item Real-time sonification provides \textbf{immediate, intuitive feedback} on integrator quality.
    \item The framework has \textbf{educational value}, allowing naive users to distinguish stable from unstable simulations.
\end{enumerate}

For Hamiltonian systems, \textbf{geometric structure trumps formal accuracy}. A second-order symplectic method outperforms a fourth-order non-symplectic method in long-time simulations.

\section*{Acknowledgment}

This work is part of the Biophysics Portfolio self-study program in computational structural biology.

\begin{thebibliography}{10}

\bibitem{karplus2002}
M. Karplus and J. A. McCammon, ``Molecular dynamics simulations of biomolecules,'' \textit{Nature Structural Biology}, vol. 9, no. 9, pp. 646--652, 2002.

\bibitem{arnold1989}
V. I. Arnold, \textit{Mathematical Methods of Classical Mechanics}, 2nd ed. New York: Springer-Verlag, 1989.

\bibitem{leimkuhler2004}
B. Leimkuhler and S. Reich, \textit{Simulating Hamiltonian Dynamics}. Cambridge: Cambridge University Press, 2004.

\bibitem{bregman1990}
A. S. Bregman, \textit{Auditory Scene Analysis: The Perceptual Organization of Sound}. Cambridge, MA: MIT Press, 1990.

\bibitem{marsden1999}
J. E. Marsden and T. S. Ratiu, \textit{Introduction to Mechanics and Symmetry}, 2nd ed. New York: Springer-Verlag, 1999.

\bibitem{hairer2006}
E. Hairer, C. Lubich, and G. Wanner, \textit{Geometric Numerical Integration}, 2nd ed. Berlin: Springer-Verlag, 2006.

\bibitem{verlet1967}
L. Verlet, ``Computer `experiments' on classical fluids. I. Thermodynamical properties of Lennard-Jones molecules,'' \textit{Physical Review}, vol. 159, no. 1, pp. 98--103, 1967.

\bibitem{skeel1997}
R. D. Skeel, G. Zhang, and T. Schlick, ``A family of symplectic integrators,'' \textit{SIAM J. Sci. Comput.}, vol. 18, no. 1, pp. 203--222, 1997.

\bibitem{swope1982}
W. C. Swope \textit{et al.}, ``A computer simulation method for the calculation of equilibrium constants,'' \textit{J. Chem. Phys.}, vol. 76, no. 1, pp. 637--649, 1982.

\bibitem{frenkel2002}
D. Frenkel and B. Smit, \textit{Understanding Molecular Simulation}, 2nd ed. San Diego: Academic Press, 2002.

\end{thebibliography}

\appendix

\section{Non-Symplecticity of Forward Euler}

For the harmonic oscillator with $\omega = 1$, the Forward Euler Jacobian is:

\begin{equation}
M = \begin{pmatrix} 1 & \Delta t \\ -\Delta t & 1 \end{pmatrix}
\end{equation}

The determinant is:
\begin{equation}
\det(M) = 1 + \Delta t^2 > 1
\end{equation}

This proves Forward Euler is non-symplectic, with phase space expansion at each step.

\section{Symplecticity of Velocity Verlet}

Velocity Verlet is a composition of shear maps:
\begin{enumerate}
    \item Kick: $v \to v + \frac{\Delta t}{2}a(x)$
    \item Drift: $x \to x + \Delta t \cdot v$
    \item Kick: $v \to v + \frac{\Delta t}{2}a(x)$
\end{enumerate}

Each shear has unit determinant Jacobian. The composition of symplectic maps is symplectic.

\end{document}
